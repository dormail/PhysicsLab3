\section{Letter encoding}
\label{sec:letter-encoding}
The letters get encoded by converting their index in a binary representation. To encode 26 letters,
the number of bits $n$ has to be so larger, such that $2^n \geq 26$, the smallest $n$ turns out to
be $5$. The whole scheme is shown in \autoref{tab:letter-encoding}, leaving only $6$ possible bits
empty.

\begin{table}
  \centering
  \caption{Binary representation of the alphabet, taken from \cite{thor_manual}.}
  \label{tab:letter-encoding}
  \sisetup{table-format=2.1}
  \begin{tabular}{c | c c c c c}
    A & 0 & 0 & 0 & 0 & 0 \\
    B & 0 & 0 & 0 & 0 & 1 \\
    C & 0 & 0 & 0 & 1 & 0 \\
    D & 0 & 0 & 0 & 1 & 1 \\
    E & 0 & 0 & 1 & 0 & 0 \\
    F & 0 & 0 & 1 & 0 & 1 \\
    G & 0 & 0 & 1 & 1 & 0 \\
    H & 0 & 0 & 1 & 1 & 1 \\
    I & 0 & 1 & 0 & 0 & 0 \\
    J & 0 & 1 & 0 & 0 & 1 \\
    K & 0 & 1 & 0 & 1 & 0 \\
    L & 0 & 1 & 0 & 1 & 1 \\
    M & 0 & 1 & 1 & 0 & 0 \\
    N & 0 & 1 & 1 & 0 & 1 \\
    O & 0 & 1 & 1 & 1 & 0 \\
    P & 0 & 1 & 1 & 1 & 1 \\
    Q & 1 & 0 & 0 & 0 & 0 \\
    R & 1 & 0 & 0 & 0 & 1 \\
    S & 1 & 0 & 0 & 1 & 0 \\
    T & 1 & 0 & 0 & 1 & 1 \\
    U & 1 & 0 & 1 & 0 & 0 \\
    V & 1 & 0 & 1 & 0 & 1 \\
    W & 1 & 0 & 1 & 1 & 0 \\
    X & 1 & 0 & 1 & 1 & 1 \\
    Y & 1 & 1 & 0 & 0 & 0 \\
    Z & 1 & 1 & 0 & 0 & 1 \\
  \end{tabular}
\end{table}

