\section{Introduction}
\label{sec:introduction}
In this report we will present an optical approach to quantum key distribution, where the BB84
protocol is reproduced using photon polarization states. \label{sec:BB84} will give a short overview
over the key exchange scheme and \autoref{sec:procedure} will contain the setup which we will use to
test the protocol. Lastly, \autoref{sec:results} will show our results, including the encyption
step.

\subsection{The BB84 protocol}
\label{sec:BB84}
The BB84 is a quantum key distribution protocol which two different bases for a two level quantum
system, to make interception by a third party impossible due to the \textit{no cloning theorem}.
Necessary for the protocol is an authenticated channel between the parties.

The two bases are not uniquely defined, however we use the two bases
\begin{align}
  S_z = \{ \ket{0}, \ket{1} \} \qquad
  S_x = \{ \ket{+}, \ket{-} \},
\end{align}
with the superposition states
\begin{align}
  \ket{+} = \frac{\ket{0} + \ket{1}}{\sqrt{2}} \qquad
  \ket{-} = \frac{\ket{0} - \ket{1}}{\sqrt{2}}.
\end{align}
These can be directly mapped to the polarization states of photons, where a conversion between these
to are rotations along the propagation axis. The operation mapping these basis towards each other is
commonly refered to as the \textit{Hadamard matrix}
\begin{equation}
  H = \frac{1}{\sqrt{2}}
  \begin{pmatrix}
    1 & 1 \\
    1 & -1 
  \end{pmatrix}.
\end{equation}
In this notation the $\ket{0}$ state is denoted as the first entry, $\ket{1}$ as the second entry of
a 2-vector. The same transformation can be realized through a $45^\circ$ rotation on the
polarization.

To send a key, between to parties A and B, the first party generates a sequence of $N$ bits
and chooses a basis to decode each bit, both being random. This generates a list consisting of
tuples. As an example for $N=5$:
\begin{align*}
  I_A = \{ (\text{bit}_i, \text{basis}_i \}
    = \{ &(1, S_x) \\
         &(0, S_z) \\
         &(1, S_z) \\
         &(0, S_x) \\
         &(1, S_z) \}
\end{align*}
These are sent accordingly to party B, which measures them in a random basis as well. If the basis
matches, B`s measurement should give the same result as what A, encoded. For different bases, the
result will be random. A possible set of bases and measured bits reads
\begin{align*}
  I_B
    = \{ &(1, \textcolor{green}{S_x}) \\
         &(1, \textcolor{red}{S_x}) \\
         &(1, \textcolor{green}{S_z}) \\
         &(0, \textcolor{red}{S_z}) \\
         &(1, \textcolor{green}{S_z}) \}.
\end{align*}
Note that while most results are deterministic, the second and fourth bit measured are random, as
A`s and B`s basis do not match for those bits.

In the last step, A and B compare their chosen bases over the authenticated channel. Both parties
discard the bits where the basis did not match; the remaining bits will form the key. In our case
that would lead to the key
\begin{equation}
  K = 111_2.
\end{equation}

To check for an eavesdropper, A and B can exchange a section of the key, to see if it matches, where
an eavesdropper would have no higher chance then guessing, i.e. $\frac{1}{2^n}$ for $n$ bits in this
test.

