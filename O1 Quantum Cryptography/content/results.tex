\section{Results}
\label{sec:results}

\subsection{Key exchange}
\label{sec:Key exchange without Eve}
The initially generated sequence and measurement basis are shown in
\autoref{tab:res:initial-sequence}. As a second step the parties compare their bases via a non
quantum channel, which has been included as the basis match column. Leaving out the eavesdropper
detection, we gained the five bit key
\begin{equation}
  \text{Key} = 00111_2.
  \label{eqn:res:key}
\end{equation}
As a basis we will continue using the bases from the key bits
\begin{equation}
  \text{Basis} = (Z, Z, Z, Z, X). 
\end{equation}

\begin{table}
  \centering
  \caption{Pseudo randomly generated bit and basis for A and B. The basis columns already contain
  the angle for encoding the bit in the A basis.}
  \label{tab:res:initial-sequence}
  \sisetup{table-format=2.1}
  \begin{tabular}{c | c | c | c | c}
    Bit & A angle & B basis & Result & Basis match \\
    \hline
    0 & 0 & 0 & 0 & Y \\
    0 & -45 & 0 & 1 & N \\
    1 & 90 & 45 & 0 & N \\
    0 & 0 & 0 & 0 & Y \\
    1 & 45 & 0 & 1 & N \\
    1 & 90 & 0 & 1 & Y \\
    0 & 0 & 0 & 1 & N \\
    1 & 90 & 0 & 1 & Y \\
    1 & 45 & 45 & 1 & Y \\
    0 & -45 & 0 & 0 & N \\
  \end{tabular}
\end{table}

\subsection{Sending a message with an eavesdropper}
\label{sec:Sending a message without an eavesdropper}
In the following part we will show the transmission of the five letter word ``KAIST`` via our
secured quantum channel. Using the encoding scheme explained in \autoref{sec:letter-encoding} 
and the encryption using a XOR operation, we get 
\begin{align}
  M_2 &= \underbrace{01010}_{K} \, 
  \underbrace{00000}_{A} \,
  \underbrace{01000}_{I} \, 
  \underbrace{10010}_{S} \, 
  \underbrace{10011_2}_{T} \\
  M_{2, \text{encr}} &= 01101 \, 00111 \, 01111 \, 10101 \, 10100_2.
\end{align}
This message is sent through a quantum channel with an eavesdropper who has to guess the basis and
measures the signal, afterwards he sends the measured quantum information to party B. The angles and
measured signals are shown in \autoref{tab:res:sending-sequence}.

\begin{table}
  \centering
  \caption{Results of sending a message through an eavesdropper (E). The basis for E has been
  pseudorandmly generated. The angle corresponds to the bit to be sent in the specific basis or the
basis of measurement. Since E does not change his basis between receiving (R) and transmitting (T),
there is only a $90^\circ$ change in the angle when the measurement yielded 1.}
  \label{tab:res:sending-sequence}
  \sisetup{table-format=2.1}
  \begin{tabular}{c | c | c | c | c | c | c}
    A bit &
    A angle &
    E R angle &
    E bit &
    E T angle &
    B basis &
    B bit \\
    \hline
    0&0&45&1&45&0&1 \\
    1&90&0&1&90&0&1 \\
    1&90&0&1&90&0&1 \\
    0&0&45&1&45&0&1 \\
    1&45&45&1&45&45&1 \\
    0&0&45&1&45&0&0 \\
    0&0&0&0&0&0&0 \\
    1&90&0&1&90&0&1 \\
    1&90&45&1&45&0&0  \\
    1&45&45&1&45&45&1 \\
    0&0&45&1&45&0&1 \\
    1&90&45&1&45&0&0  \\
    1&90&45&1&45&0&1  \\
    1&90&45&0&-45&0&0 \\
    1&45&45&1&45&45&1 \\
    1&90&0&1&90&0&1 \\
    0&0&0&0&0&0&0 \\
    1&90&0&1&90&0&1 \\
    1&90&45&1&45&0&1  \\
    1&90&45&0&-45&0&0 \\
    1&45&45&1&45&45&1 \\
    1&90&0&1&90&0&1 \\
    0&0&0&0&0&0&0 \\
    1&90&0&1&90&0&1 \\
    0&0&0&0&0&0&0 \\
    1&45&45&1&45&45&1 \\
    1&90&45&0&-45&0&0 \\
    0&0&0&0&0&0&0 \\
    1&90&45&0&-45&0&1 \\
    0&0&0&0&0&0&0 \\
    0&-45&0&0&0&45&1  \\
  \end{tabular}
\end{table}

This gives a detected message at Bob`s party
\begin{align}
  M_{2, \text{encr}} &= 11111 \, 
                        00101 \,
                        10101 \,
                        10101 \,
                        00101 \\
  M_{2} &= \underbrace{11000}_{Y} \, 
           \underbrace{00010}_{C} \,
           \underbrace{10010}_{S} \,
           \underbrace{10010}_{S} \,
           \underbrace{00010}_{C}
  \label{eqn:final_message}
\end{align}
