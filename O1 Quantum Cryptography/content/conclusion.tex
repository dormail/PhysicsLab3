\section{Conclusion}
\label{sec:conclusion}
In this experiment a quantum key exchange and transmission have been implemented using optical
polairzation states.

An eavesdropper has been detected through a low fidelity of the transmission, leading to a channel
indistiguishable from random bits.

Important to note is that this setup can not be used in the real world, since the no-cloning
theorem only works for individual photons. Since an optical beam contains an abundance of photons,
an eavesdropper can perform a measurement on only a small amount photons. With multiple
measurements, he can even guess the basis correctly then. An ideal implementation therefore would
use one photon per (qu-)bit, such that an eavesdropper can not do partial measurements.

Furthermore, this protocol can be improved by introducing parity bits. As it can be seen in
\autoref{tab:letter-encoding}, a measurement of $11010_2$ and above would leed to undefined
decodings. Even though it did not appear in our measurement, on average $18.75\%$ of bit signal
could not be decoded to letters. This error detection can be imporoved by introducing parity bits or
check sums, leading to higher detecting rates with minimal overhead.

Lastly, tests can be done on the actual key exchange with an eavesdropper, 
where the BB84 is supposed to fail due to the disturbance because of the eavesdropper performing
measurements.

