\section{Finding $T_C$}
\label{sec:Finding $T_C$}
To find the critical temperature $T_C$ based on our experimental data, the curve fit from the scipy
library will be used, hence the challenge is finding an appropriate fit function and initial
condition.

The critical behaviour is modelled through an arcus tangens function
\begin{equation}
  R_\text{crit}(T) = A \arctan\left( (T - T_C) \cdot s\right)
\end{equation}
where $T_C$ corresponds to the critical temperature, $s$ is a parameter controlling the abruptness
of the critical behaviour when $T \approx T_C$. $A$ refers to the amplitude of the critical jump.
Since $R$ shows a linear behaviour when the temperature is not close to the critical point, we add
a linear function
\begin{equation}
  R_\text{lin}(T) = m\cdot T + b,
\end{equation}
where $m$ and $b$ are simple coefficients with no deeper meaning for our work. This gives the fit
function
\begin{equation}
  R(T) = m\cdot T + b + A \arctan\left( (T - T_C) \cdot s\right).
\end{equation}
The initial values of the parameters for the optimizer have been estimated visually and are given in
\autoref{tab:fitparams}.

\begin{table}
  \centering
  \caption{Initial values $p_0$ of the fit parameters for the fit function explained in
  \autoref{sec:Finding $T_C$}.}
  \label{tab:fitparams}
  \sisetup{table-format=2.1}
  \begin{tabular}{c c c}
    Parameter & $p_0^\text{cooling}$ & $p_0^\text{heating}$ \\
    \hline
    $m$ & 0.0001 & 0.0001 \\
    $b$ & 0 & 0 \\ 
    $A$ & 0.001 & 0.001 \\
    $s$ & 100 & 100 \\
    $T_C$ & 70 & 100
  \end{tabular}
\end{table}

