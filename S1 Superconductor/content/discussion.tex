\section{Discussion}
\label{sec:discussion}
The main results of this work are the two values for the critical temperature of YBCO:
\begin{align}
  T_C^\text{cool} &= (83.729 \pm 0.011) \, \text{K}, \\
  T_C^\text{heat} &= (90.47 \pm 0.14) \, \text{K}.
\end{align}
A corresponding value from the literature is well known to be $T_C = \SI{95}{K}$ \cite{19971173}.

Remarkable about our pair of values is that each value onto its own shows only a small uncertainty
but deviate from each other by $\SI{6.741}{K}$ or 48 times the bigger standard deviation. This can
be explained by a delay between measuring a temperature in the chamber and the sample acutally
reaching that temperature. Therefore, the measured temperature can be beyond $T_C$ while the
physical structure of the sample did not reach the critical point yet.

There is a significant deviation between our results and the literature value, which can be easily
explained by imperfect ratios between the raw products. Furthermore, mishandling of the sample
during the synthesis stage can lead to an imperfect atomic structure. 

Lastly, it is remarkable how little relative change in electrical resistance has been measured when
crossing the critical point. This can be explained by the inner resistance of the cables and probes.

The absence of the Meissner-Ochsenberg effect is interesting considering we still measured a drop in
electrical resistance. This can be due to our sample being smaller then the given one, leading to
a smaller force against the neodymium magnet. Furthermore, seeing only a small drop in electrical
resistance, there could be an imperfect superconducting state, where only parts of the sample built
up the correct atomic structure. This could be due to improper mixing and grinding during the
synthesization.

