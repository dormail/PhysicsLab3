\section{Results}
\label{sec:results}

\subsection{Observation of the Meissner-Ochsenberg effect}
\label{sec:Observation of the Meissner-Ochsenberg effect}
After completing the synthesis, we test the properties of our YBCO sample. Qualitatively this has
been done by testing the Meissner-Ochsenberg effect as it has been introduced in
\autoref{sec:intr:meissner}. Although we could not reproduce it with our sample, we were able to
bring a neodymium magnet to levitate on antoher given YBCO sample, by cooling the later with liquid
nitrogen.

\subsection{Measuring low temperature resistance}
\label{sec:Measuring low temperature resistance}
The sample with the measuring probes and circuit has been cooled down to $\SI{50}{K}$. A computer
attached to the setup up measured the temperature and resistance once every 2.5 seconds. The
function $R(T)$ is plotted for both runs in \label{fig:R-T}.
\begin{figure}
  \centering
  \includegraphics[width=.5\textwidth]{build/resistance-temperature.pdf}
  \caption{Sample plot for the conductivity of a superconductor.}
  \label{fig:R-T}
\end{figure}
To calculate the critical temperature we used the curve fit as introduced in 
\autoref{sec:Finding $T_C$}. This gave two fit parameters
\begin{align}
  T_C^\text{cool} &= (83.729 \pm 0.011) \, \text{K}, \\
  T_C^\text{heat} &= (90.47 \pm 0.14) \, \text{K}.
\end{align}
The uncertainties have been calculated using the covariance matrix of the curve fit. The fits and
data are shown together in \autoref{fig:fits}.
\begin{figure}
     \centering
     \begin{subfigure}[b]{0.5\textwidth}
         \centering
         \includegraphics[width=\textwidth]{build/fit_cooling.pdf}
         \caption{Cooling}
         \label{fig:fit cooling}
     \end{subfigure}
     \\
     \begin{subfigure}[b]{0.5\textwidth}
         \centering
         \includegraphics[width=\textwidth]{build/fit_heating.pdf}
         \caption{Heating}
         \label{fig:fit heating}
     \end{subfigure}
     \caption{Resistance data and curve fits for cooling (a) and  heating (b).}
        \label{fig:fits}
\end{figure}

