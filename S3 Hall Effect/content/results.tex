\section{Results}
\label{sec:results}

\subsection{Hall voltage as a function of current}
\label{sec:res-a}
According to \autoref{eqn:intr:U_H}, we expect a linear behavior for the Hall voltage $U_\text{H}$ 
when the current $I$ flowing through the sample is varied. Furthermore, there is also linear
dependence on the magnetic field, which in turn is proportional to the current flowing through the
coils. In this section the results of \autoref{sec:procedure:a} shall be presented.

For each run we create a linear regression line according to the formula
\begin{equation}
  f(x; m, b) = m\cdot x + b
\end{equation}
where $m$ and $b$ are variable parameters. The function will be fitted to each data line using
scipy.optimize. For the three cases n-dotted, p-dotted, non-dotted we did three runs with three
different magnetic field strengths, giving nine combinations. These will be listed in the folowing.

For the n-dotted we got
\begin{align*}
  \SI{3}{A} \quad m &= (-0.0406 \pm 0.0032) \frac{\text{V}}{\text{A}} \\
                  b &= (-0.00061 \pm 0.00008) \si{V} \\
  \SI{4}{A} \quad m &= (-0.2343 \pm 0.0032) \frac{\text{V}}{\text{A}} \\
                  b &= (-0.00055 \pm 0.00008) \si{V} \\
  \SI{5}{A} \quad m &= (-0.4100 \pm 0.0032) \frac{\text{V}}{\text{A}} \\
                  b &= (-0.00053 \pm 0.00008) \si{V}.
\end{align*}
For the p-dotted
\begin{align*}
  \SI{3}{A} \quad m &= (0.9227 \pm 0.0016) \frac{\text{V}}{\text{A}} \\
                  b &= (-0.000367 \pm 0.000035) \si{V} \\
  \SI{4}{A} \quad m &= (1.1921 \pm 0.0016) \frac{\text{V}}{\text{A}} \\
                  b &= (-0.000387 \pm 0.000035) \si{V} \\
  \SI{5}{A} \quad m &= (1.4336 \pm 0.0016) \frac{\text{V}}{\text{A}} \\
                  b &= (-0.000370 \pm 0.000035) \si{V}.
\end{align*}
For the undotted case
\begin{align*}
  \SI{3}{A} \quad m &= (-0.006 \pm 0.021) \frac{\text{V}}{\text{A}} \\
                  b &= (-0.00075 \pm 0.00022) \si{V} \\
  \SI{4}{A} \quad m &= (-0.004 \pm 0.021) \frac{\text{V}}{\text{A}} \\
                  b &= (-0.00078 \pm 0.00022) \si{V} \\
  \SI{5}{A} \quad m &= (-0.012 \pm 0.021) \frac{\text{V}}{\text{A}} \\
                  b &= (-0.00075 \pm 0.00022) \si{V}.
\end{align*}
The measured data and the linear regressions are shown in \autoref{fig:res-a}.

\begin{figure}
     \centering
     \begin{subfigure}[b]{0.4\textwidth}
         \centering
         \includegraphics[width=\textwidth]{build/V-A-n.pdf}
         \caption{n-dotted medium}
         \label{fig:a-n}
     \end{subfigure}
     \hfill
     \begin{subfigure}[b]{0.4\textwidth}
         \centering
         \includegraphics[width=\textwidth]{build/V-A-p.pdf}
         \caption{p-dotted medium}
         \label{fig:a-p}
     \end{subfigure}
     \hfill
     \begin{subfigure}[b]{0.4\textwidth}
         \centering
         \includegraphics[width=\textwidth]{build/V-A-e.pdf}
         \caption{undotted medium}
         \label{fig:a-e}
     \end{subfigure}
        \caption{Hall Voltage as a function of current flowing through sample. For three different
        magnetic field strengths indicated by the coil voltages. Note that different scales for the
      different graphs are being used.}
        \label{fig:res-a}
\end{figure}

\subsection{Hall voltage as a function of the magnetic field}
\label{sec:res-b}
In this part we show the result when measuring the dependence between the Hall voltage and the
magnetic field. We performed three runs for each dotting to smoothen the results, which are shown 
in \autoref{fig:res-b}. 

Since there were to many data points for a curve fit, the linear regression function from scipy has
been used in this part.

\begin{figure}
     \centering
         \includegraphics[width=.4\textwidth]{build/V-B.pdf}
        \caption{Hall Voltage as a function of the applied magnetic field, for n-dotted, p-dotted
        and undotted Germanium.}
        \label{fig:res-b}
\end{figure}
