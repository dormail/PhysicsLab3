\section{Discussion}
\label{sec:discussion}
While the results all show the key features, the qualitative analysis shows some errors. The
qualitative features are
\begin{itemize}
  \item the linear dependence between magnetic field and Hall voltage,
  \item the linear dependence between sample current and Hall voltage,
  \item the polarization of $U_\text{H}$ depending on the charge of the charge carriers.
\end{itemize}
While for each individual measurement the linear dependecies are correct, as shown by small
relative uncertainties in the linear regressions which are all below $10\%$ except for the undotted
case. The undotted case shows a constant zero and is essentially noise, so uncertainties can be
higher.

Where the linear dependence is broken how ever, is when comparing different magnetic field in
\autoref{sec:res-a}. While the slope $m$ should be proportional to the field strength $B$, the
ratio between different slopes is not close to the ratio between field strength. An example
calculation for the n-dotted measurement:
\begin{equation}
  \frac{m(\SI{5}{A})}{m(\SI{4}{A})} = \frac{0.41}{0.2343} \approx 1.75 > \frac{5}{4} = 1.25
\end{equation}
However the ratio is closer to the theoretical value for the p-dotted sample:
\begin{equation}
  \frac{m(\SI{5}{A})}{m(\SI{4}{A})} = \frac{1.4336}{1.1921} \approx 1.203
\end{equation}
Especially the $\SI{3}{A}$ measurement for the n-dotted Germanium shows a suspicously low slope.

A large source for error is electornic noise. For the undotted Germanium, where you would expect no
Hall effect, voltages of over $\SI{0.003}{V}$ have been measured, which corresponds to more than
$10\%$ of the absolute values measured in the other runs. 

This uncertainty explains the large deviation from the expected slope with low currents. Because the
Hall voltage is low, the noise make it even harder to determine a voltage from the background,
artificially lowering the slope.

This error can be mittigated with more sophisticated compensation and better shielding of cables.

