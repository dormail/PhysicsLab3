\section{Discussion}
\label{sec:discussion}
As presented in \autoref{sec:results}, in this work we obtained the values
\begin{align}
  c_\text{Air} &= (2.97 \pm 0.04) 10^8 \, \si{m/s} \\
  c_\text{resin} &= (1.9 \pm 0.1) \cdot 10^8 \, \si{ms^{-1}}.
\end{align}
The first one deviates by $0.93 \%$ from the literature value, though the deviation is smaller then
its uncertainty.

Since the exact composition of the resin is unknown, we can only compare our results to the ones
given in the manual. For the refractive index, the deviation is $1.06 \%$, and the result for the
speed of light in resin deviates by $1.6 \%$, showing the highest inaccuracy among our results. For
the two results in the case of resin, the deviation is smaller then the uncertainty, so there is no
general disagreement between our results and the manual.

There are still sources for error which leave room for future improvement. While our technique of
using a phase difference of $\pi/2$ gave good results, it is harder for the experimentalist to
accurately set it as the Lissajous figure for $\Delta \phi = \pi/2$ is a circle or elliptical. This
results in potentially single sided biases for $\Delta x$. To overcome this error source, a longer
experimental setup is needed and a stronger light source, as the used laser lost its sharpness after
$l \gtrapprox \SI{1}{m}$.

A second source of error are the lenses used to focus the beams, as we modified it only minimally, a
better tuning could result in sharper Lissajous figures.

For the resin specifically, the probe was of bad quality. On each surface were scratches leading to
diffraction when the light beam enters and leaves. The precision could be greatly enhanced with a
better quality for the probe.

