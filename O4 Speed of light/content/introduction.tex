\section{Introduction}
\label{sec:introduction}

\subsection{History}
\label{sec:history}

\subsection{Theory}
\label{sec:theory}

\subsubsection{Deriving the speed of light from Maxwell`s equations}
The speed of light as a fundamental constant can be derived from Maxwell`s equations
\cite{LabInstructions}
\begin{equation}
  \nabla \times \mathbf{E} = - \frac{\partial \mathbf{B}}{\partial t}, \quad
  \nabla \times \mathbf{B} = \mu_\text{abs} \epsilon_\text{abs} 
                              \frac{\partial \mathbf E}{\partial t} .
\end{equation}
where $\mu_\text{abs}$ and $\epsilon_\text{abs}$ are the absolute permeability and permittivity of
the medium. They can be related to the relative and vacuum constant for each size like
\begin{align}
  \epsilon_\text{abs} &= \varepsilon \varepsilon_0, \\
  \mu_\text{abs} &= \mu_r \mu_0.
\end{align}
In these equations $\varepsilon_0$ is called the vacuum permittivity and $\varepsilon$ is a
medium-specific \textit{relative permittivity}. In a similar fashion, $\mu_0$ is the vacuum
permeability and $\mu_r$ the \textit{relative permeability}.

By taking the first time derivative of either equation and plugging the result into the other, one
can get a wave equation after the double curl has been reduced to a laplace operator. From the wave
equation a phase velocity of
\begin{equation}
  c = \frac{1}{\sqrt{\varepsilon \varepsilon_0 \mu_0 \mu_r}}
\end{equation}
becomes apparent.

Since $\varepsilon_0$ and $\mu_0$ are physical constants they can be measured and their values are
well known to be \cite{muzero,epsilonzero}
\begin{equation}
  \varepsilon_0 = \SI{8.854e-12}{Fm^{-1}}, \quad
  \mu_0 = \SI{1.256e-6}{NA^{-2}}.
\end{equation}

With the relative permeability and relative permittivity of a medium we can define the
\text{refractive index}
\[
  n = \sqrt{\varepsilon \mu_r}.
\]
$n$ depends, like $\mu_r$, on the frequency of the light because of dispersion.


