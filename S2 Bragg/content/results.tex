\section{Results}
\label{sec:results}

\subsection{Determining the wavelength of the x-ray source}
\label{sec:res:wavelength}
The wavelength has been determined in two runs, the first being a test if the machine has been set
up properly and the second going over the meaningful range of $\theta$. The raw data for both runs
is shown in \autoref{fig:nacl}. The peak has its center at
\[
  \theta = 5.5^\circ,
\]
using the Bragg condition the wavelength turns out to be
\begin{equation}
  \lambda_1 = 2d \sin(\theta) = \SI{54.06}{pm}.
\end{equation}
Here we assumed a first order reflection and a spacing $d = \SI{282.01}{pm}$ as given in the manual
\cite{leybold_manual1}.

There is also a slight second order reflection visible at $\theta = 13.1^\circ$, where one can
calculate
\begin{equation}
  \lambda_2 = \SI{63.92}{pm}.
\end{equation}

\begin{figure}
  \centering
  \begin{subfigure}[b]{0.45\textwidth}
    \centering
    \includegraphics[width=\textwidth]{build/NaCl2.pdf}
    \label{fig:nacl_calib}
  \end{subfigure}
  \\
  \begin{subfigure}[b]{0.45\textwidth}
    \centering
    \includegraphics[width=\textwidth]{build/NaCl1.pdf}
    \label{fig:nacl}
  \end{subfigure}
  \caption{Measuring the x-ray reflection of NaCl in two runs.}
  \label{fig:nacl}
\end{figure}

\subsection{Determining the lattice constant of LiF and Ti}
\label{sec:res:d}
The method used in the last section can now be used to measure lattice constants. For the
calculations we will use the spectral wavelengths given in the manual. 

For both materials we did one full run after multiple calibration runs. The first runs only gave bad
results with no peaks visible. \autoref{fig:lattice_constants} shows the reflection intensity for
LiF and Ti.

We found the glance angles to be
\begin{align}
  \lambda^\text{Ti} &= 8.1^\circ \\
  \lambda^\text{LiF} &= 7.9^\circ.
\end{align}
Assuming first order reflections, we get the lattice constants
\begin{align}
  &K_\alpha: & a_0^\text{Ti} &= \SI{517.15}{pm} & a_0^\text{LiF} &= \SI{504.47}{pm} \\
  &K_\beta: & a_0^\text{Ti} &= \SI{459.02}{pm} & a_0^\text{LiF} &= \SI{447.76}{pm} \\
\end{align}
since we only saw one peak we can not savely determine wether it is a $K_\alpha$ or $K_\beta$ beam.

\begin{figure}
  \centering
  \begin{subfigure}[b]{0.45\textwidth}
    \centering
    \includegraphics[width=\textwidth]{build/LiF.pdf}
    \label{fig:lif}
  \end{subfigure}
  \\
  \begin{subfigure}[b]{0.45\textwidth}
    \centering
    \includegraphics[width=\textwidth]{build/Ti.pdf}
    \label{fig:ti}
  \end{subfigure}
  \caption{Measuring the x-ray reflection of LiF (top) and Ti (bottom) to
  determine their lattice constants.}
  \label{fig:lattice_constants}
\end{figure}
