\section{Introduction}
\label{sec:introduction}
In this experiment we built a Helium-Neon (HeNe) laser using simple optical elements.

\subsection{Optical cavities}
\label{sec:Optical cavities}
The key role in creating a laser is filled by the optical cavity, a set of mirrors which acts as a
resonator for light in it. Optical cavities can be classified by the mirrors used and the shape of
them.

A simple cavity can be plane-parallel, two flat mirrors directly pointed at each other. Since it is
hard to align, we will use two concave mirrors pointing at each other. Depending on the ratio
between their curvature radii $R_i$ and the distance $d$ between the mirrors, there are two special
cases
\begin{itemize}
  \item concentric spherical $R_i = \frac{d}{2}$: A focal point in the middle of the cavity is
    created,
  \item confocal $R_i = d$: The beam is the most focussed at the mirrors.
\end{itemize}
Another interesting feature is breakdown of the cavity for $R_i > d$. We will work in the regime between 
these two cases, so $R_i \leq d \leq 2R_i$. $R_i=\SI{700}{mm}$ is fixed since we only have one pair 
of mirrors.

