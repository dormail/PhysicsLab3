\section{Results}
\label{sec:results}
Following the steps provided in the last section, we achieved to get a laser oscillation. We were
able to drag out the mirrors up to approximately $\SI{135}{cm}$, only $\SI{5}{cm}$ below the
theoretical limit of $\SI{140}{cm}$ due to the curvature radii of the mirrors. This laser has been
analysed further, which we will present in this section.

\subsection{Spectrum of laser}
\label{sec:Spectrum of laser}
Using a spectrometer, we measured the optical spectrum of the laser, which is shown in
\autoref{fig:spectrum}. There is a clear peak at
\begin{equation}
  \lambda = \SI{634}{nm}.
\end{equation}
To clean the spectrum from background, we used the procedure explained in \autoref{sec:cleaning}.

\begin{figure*}
  \centering
  \includegraphics{build/spectrum.pdf}
  \caption{Spectrum of the He-Ne laser, after clean up. The highest intensity has been measured at $\SI{634}{nm}$
  and the data has been normalized to this value.}
  \label{fig:spectrum}
\end{figure*}

\subsection{Dependence on tube current}
\label{sec:Dependence on tube current}
After finding the peak in the spectrum, we can measure the dependency of the intensity on the
current flowing through the tube. We used the full range the power suply offered, going from
$\SI{5}{mA}$ to $\SI{6.5}{mA}$ in $\SI{0.1}{mA}$ steps. For each data point, we recorded the
relative intensity at the peak found in the step before. The results are shown in
\autoref{fig:intensity}. A clear trend downwards can be seen when going above $\SI{5.8}{mA}$.

\begin{figure*}
  \centering
  \includegraphics{build/intensity.pdf}
  \caption{Laser intensity measured vs. tube current.}
  \label{fig:intensity}
\end{figure*}
