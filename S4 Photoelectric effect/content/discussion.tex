\section{Discussion}
\label{sec:discussion}
A qualitative analysis based on \autoref{fig:photocurrent} yields the following results:
\begin{itemize}
  \item Our data shows linear behavior in the negative regime,
  \item the positive range behaves as expected from \autoref{fig:intr:photocurrent},
  \item in the range $[0, \SI{10}{V}]$ our results only show a small deviation form the
    data given to us,
  \item while our measurements show small relative uncertainties below $5\%$, the measurements in the
    negative regime disagree with each other and have high relative uncertainty of up to $8.96\%$.
\end{itemize}
Especially the different behavior for $V_\text{C}<0$ and $V_\text{C}>0$ is an interesting feature
which we would like to explore in this discussion.

A probable source of error is mishandling or malfunctioning operating units. The software
controlling the photocell turned out to be rather unintuitive and sometimes gave errors
mid-measurement.

Futhermore, since we did not assemble the setup ourselves a mistake in the preperation can not be
excluded either. These mistakes can effect the UV light, spacing between sample and collector or the
electronic connections connecting them. The components mentioned and their quality have strong impact
on the measurement and can lead to difference between an accelerating current and deccelerating
current, which is what we observed.

A further source of error was the reading of the photo current. After changing $V_\text{C}$, the
photo current changed as well, but showed ``relaxing`` behavior, going back towards the value which
it had before adjusting the voltage. For consistency we took the first value readable after setting
$V_\text{C}$. Even with this strategy it was hard to reproduce our own results and the short time
frame to read values is prone to human error.

