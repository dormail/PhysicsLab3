\section{Procedure}
\label{sec:procedure}
A sketch of the setup is given in \autoref{fig:ExpSetup}. Both the figure and the following steps
are taken from the manual given out for this experiemtn \cite{LabInstructions}. 

The key components are:
\begin{itemize}
  \item The photocell with an UV light source,
  \item power supply,
  \item the picoamperemeter, which can measure currents in the order of $10^{12}$ Ampere.
\end{itemize}
The actual setup in the laboratory is more complicated, luckily the preperation has been done in
advance, e.g. preparing sample and collector in a vacuum chamber. Controlling the light source
happens through a computer with the software Elux-Control.

The power supply creates a voltage $V_\text{C}$ between sample and collector, which can be used as
described \autoref{sec:intr:colletor_voltage}. 
The photocurrent can be read from the ammeter. As a range for $V_\text{C}$ we
use $\SI{-30}{V}$ to $\SI{30}{V}$ with a step size of $\SI{1}{V}$. Since the power supply can not be
set to 0, there is a data point for $\pm \SI{0.01}{V}$ each and negative voltages have to be
constructed by changing the ports on the supply.

